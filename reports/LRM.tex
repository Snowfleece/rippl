\documentclass[5pt]{article}
\usepackage[utf8]{inputenc}
\usepackage{multicol}
\usepackage[rgb,dvipsnames]{xcolor}
\usepackage{amsmath}
\usepackage{siunitx}
\usepackage{setspace}
\usepackage{pxfonts}
\usepackage{tabto}
\usepackage{listings}
\usepackage[margin=1.8in]{geometry}
\addtolength{\topmargin}{-.5in}
\addtolength{\textheight}{0.5in}
\usepackage{textcomp}
\newcommand{\sq}{\textquotesingle}
\usepackage[english]{babel}
\usepackage{graphicx}
\usepackage{float}
\usepackage[colorinlistoftodos]{todonotes}

\definecolor{mygrey}{rgb}{0.5,0.5,0.5}
\lstdefinelanguage{rippl}{
    language=Python,
  keywords={ over, let, in, fun, if, then, else, and, or, not, true, false, 
  head, tail, int, bool, float, char, cons, first, sec, len, cat, main},
  morecomment = [l]{\#},
  morecomment = [n]{\{\-}{\-\}},
}
\lstset{
         numbers=left,
         language=rippl,
         showstringspaces=false,
         firstnumber=1,
       numberfirstline=true,
       numberstyle=\ttfamily\scriptsize,
     xleftmargin=5.0ex,
         basicstyle=\ttfamily\small,
         tabsize=4,
         commentstyle=\color{mygrey},
         keywordstyle=\bfseries,
         deletekeywords={map}
 }

\begin{document}


\begin{titlepage}\centering
\vspace*{\fill}
\Huge RIPPL \\
\LARGE Recursively Inferred Pure functional Pythonic Language \\
\vspace*{\fill}
\large Da Hua Chen (Riddler), Hollis Lehv (Gallbladder),\\Amanda Liu (Language Yoda), Hans Montero (Prime Minister) \\
put logo here
\end{titlepage}

\begin{titlepage}
\newcommand{\HRule}{\rule{\linewidth}{0.5mm}} % Defines a new command for the horizontal lines, change thickness here
\begin{center} 
 
\includegraphics[scale=.9]{rgu.png}\\[1cm] 

\HRule \\[0.4cm]
{ \Huge \bfseries Rippl}\\[0.2cm]
\HRule \\[0.8cm]

\spacing{1.5} \textsc{\LARGE  Recursively Inferred Pure functional \\ Programming Language}\\[1.5cm]

\large \emph{Riddler:}
Da Hua \textsc{Chen}\\
\large \emph{Gallbladder:}
Hollis \textsc{Lehv}\\
\large \emph{Language Yoda:}
Amanda \textsc{Liu}\\ 
\large \emph{Prime Minister:}
Hans \textsc{Montero}\\
\vfill % Fill the rest of the page with whitespace
\end{center} 
\end{titlepage}


\pagebreak
\tableofcontents
\pagebreak
\section{Overview}
Rippl (Recursively Inferred Pure functional Pythonic Language) is a functional language that leverages the safety and elegance of pure languages like Haskell with the intuitive syntax of Python. With list comprehensions, lazily evaluated infinite lists, a strong static type system implementing Hindley-Milner style inference, higher-order functions, and simple syntax, Rippl sets out to provide a clean and intuitive functional programming experience like no other language can.

\section{Type System}
Rippl has primitive types, higher-order types, and arrow types.

\subsection{Primitive Types}
The primitive types in Rippl are $\texttt{int}$, $\texttt{float}$, $\texttt{bool}$, and $\texttt{char}$.
\subsubsection{Integers}
\texttt{int} is the signed integer type. An integer literal may be preceded by a unary minus (\texttt{-}) to denote a negative integer. Unary plus (\texttt{+}) is not supported. Integers are written in base 10 (decimal) and cannot be written in other bases (e.g. binary, octal, hexadecimal).

\subsubsection{Floating-point Numbers}
\texttt{float} is the floating-point numerical type. A float literal may be preceded by a unary minus (\texttt{-}). Unary plus (\texttt{+}) is not supported. Float literals must be written with a decimal point and at least one digit to the left and right of the decimal point (e.g. \texttt{-4.2}, \texttt{10.0}). This format was chosen to avoid ambiguity with integer literals when used with the list range operator (\dots). Numbers not written in this format, e.g. \texttt{6.}, \texttt{.53}, and \texttt{-.0}, are not valid floats. Floating-point numbers written in scientific notation, e.g. \texttt{1.37e15}, are also not supported.
\subsubsection{Booleans}
\texttt{bool} is the Boolean type. There are two Boolean literals, \texttt{true} and \texttt{false}.
\subsubsection{Characters}
\texttt{char} is the character type. A character literal is written as a single character in single quotes e.g. \texttt{\sq A\sq}, \texttt{\sq !\sq}, \texttt{\sq+\sq}. Special characters, such as the newline character \sq\texttt{\string\n}\sq, are escaped with a backslash ($\backslash$).

\subsubsection{Summary of Primitive Types}
\begin{multicols}{3}
$Type$ \\
\hspace*{5mm} \texttt{int} \\
\hspace*{5mm} \texttt{float} \\
\hspace*{5mm} \texttt{bool} \\
\hspace*{5mm} \texttt{char} \\
\columnbreak \\
$Size$ \\
4 bytes \\
8 bytes \\
1 byte \\
1 byte \\
\columnbreak \\
$Examples$ \\
\texttt{0, 37, -2, 2019} \\
\texttt{0.0, -6.9, 3.1415926} \\
\texttt{true, false} \\
\texttt{\sq\#\sq, \sq x\sq, \sq D\sq, \sq \string\n \sq }  \\
\end{multicols}

\subsection{Higher-Order Types}
Higher-order types are represented as type constructors that are parametrically polymorphic in terms of type variables. The constructors for these types can be viewed as functions that take in a proper type as an argument and return a new type abstracted over the argument. The higher-order types in Rippl are lists, tuples, and the \texttt{maybe} sum type, which comprises \texttt{none} and the \texttt{just} constructor.
\subsubsection{Lists}
Lists are a first-order type and all elements of a list must have the same type. This means that a function that acts on lists need not concern itself with the type of the elements of the list. Such a function is said to be parameterized by the type of the elements in the list. All list operators in Rippl (see section 3.5.5) act on lists of arbitrary type. \\\\
Strings are handled and represented internally as lists of \texttt{char}, but
string literals may be written in the usual sugared representation with double quotes e.g. \texttt{"Hello, World!"}.\\\\
Lists as well as list types are delimited by brackets, and list elements are separated by commas (e.g. $\texttt{[0,3,1]}$, which has type \texttt{[int]}, $\texttt{[\sq R\sq,\sq i\sq,\sq p\sq,\sq p\sq,\sq l\sq]}\Leftrightarrow \texttt{"Rippl"}$, which has type $\texttt{[char]}$).

\subsubsection{Tuples}
Tuples are also a first-order type, parametrically polymorphic in the types of its elements. A tuple must contain exactly two elements, but each of them can be of any type. Tuples as well as their types are delimited by parentheses, and elements are separated by commas (e.g. \texttt{("PLT",4118)}, which has type \texttt{([char],int)}).

\subsubsection{\texttt{maybe} Type}
The \texttt{maybe} type offers a tool to handle errors without side effects by representing an optional value. In the case that a computation cannot return a valid value, it may return the \texttt{none} constructor of the \texttt{maybe} type. Otherwise, it can return a proper value wrapped in the \texttt{just} constructor. Since the \texttt{just} can wrap any type, the \texttt{maybe} is polymorphic.

\subsection{Arrow Types}
An arrow type is a sequence of types separated by an arrow (\texttt{->}) and represents the type of a function or operator. %TODO
\begin{lstlisting}[language=rippl]
sum_tup tup = (first tup) + (sec tup)
# sum_tup :: (int, int) -> int

apply_sec tup f = f (sec tup)
# apply_sec :: (a, b) -> b -> c -> c
\end{lstlisting}
The definition of the \texttt{sum\_tup} function performs a established computation on its argument, which has a value. On the other hand, the \texttt{apply\_sec} function takes in an arrow type argument, which means it can be applied to its other argument.

\subsection{Type Annotations and Inference}
Rippl supports type inference through an implementation of the Hindley-Milner type inference system over the core syntactic language constructs of lambda abstractions, let-bindings, application, and variable names, as well as the additional language construct of if-then-else expressions. \\

\noindent By performing type inference in an environment where literals and language native operators (see section 3.5) are bootstrapped  with a particular type, types of expressions can be inferred and checked in a complete and decidable manner. These types signatures may be concrete types like the ones below.

\begin{lstlisting}[language=rippl]
sum_ints :: int -> int -> int

count_chars :: [char] -> int

is_mutually_prime :: int -> int -> bool
\end{lstlisting}

\noindent However, these inferred types may also be parametrically polymorphic. These polymorphic types are represented as type variables like the ones below. \\

\begin{lstlisting}[language=rippl]
identity :: a -> a

len :: [a] -> int

empty_list :: [a]

list_map :: a -> b -> [a] -> [b]
\end{lstlisting}

\noindent Rippl also allows programmers to provide their own type annotations for function definitions. This is done using the Haskell-like syntax shown in the previous examples, that involves specifying the variable name followed by its full curried type signature, separated by a double colon "\texttt{::}". Type annotations are optional, but when they are provided, type-checking is performed to make sure the user-annotated types are at most as generalized as the inferred type. In other words, the type annotation must be a subtype of the inferred type. \\

\noindent The following function definition is an example of how an inferred type and annotated type can type-check directly if they are equal. \\

\begin{lstlisting}[language=rippl]
succ :: int -> int
succ n = n + 1       # inferred type is int -> int
\end{lstlisting}

\noindent The following type annotation would yield a typing error since the inferred type and annotated type contradict each other. This is due to the built in type of the integer addition and division operators and the fact that Rippl doesn't perform any implicit type promotions. \\

\begin{lstlisting}[language=rippl]
avg :: int -> int -> float
avg x y = (x + y) / 2   # inferred type is int -> int -> int
\end{lstlisting}

\noindent The following function definition and annotation also type-check correctly since the inferred type that is parametrically polymorphic with type variable \texttt{a} can be properly concretized by substituting \texttt{a} with the concrete type \texttt{[int]} to match the user-annotated type. \\

\begin{lstlisting}[language=rippl]
nest_int_list :: [int] -> [[int]]
nest_int_list l = l cons []     # inferred type is a -> [a]
\end{lstlisting}

\noindent The following function definition is an example of a type annotation that won't type-check with the inferred type. The inferred type enforces that the return type of \texttt{identity} be the same as its argument, so it's polymorphic in one type variable \texttt{a}. This is a stronger type restriction than the annotated type which states that the function is polymorphic and may return a type different than that of its argument. Since the annotated type must be a subtype of the inferred type, this would yield a type error. \\

\begin{lstlisting}[language=rippl]
identity :: a -> b
identity x = x          # inferred type is a -> a
\end{lstlisting}
\pagebreak

\section{Grammar and Syntax}
\subsection{Semantics}
The following grammar represents a high-level overview of Rippl semantics.
\begin{multicols}{2}
\noindent $\mu \ ::= $ \\
\hspace*{10mm} $| \quad \texttt{main } e = e$ \\

\noindent $e \ ::= $ \\
\hspace*{10mm} $| \quad c$ \\
\hspace*{10mm} $| \quad x$ \\
\hspace*{10mm} $| \quad e \ e$ \\
\hspace*{10mm} $| \quad \texttt{fun } x \texttt{ -> } e$ \\
\hspace*{10mm} $| \quad \texttt{let }x = e \texttt{ in } e$ \\
\hspace*{10mm} $| \quad \texttt{if } e \texttt{ then } e \texttt{ else } e$ \\
\hspace*{10mm} $| \quad \gamma$ \\

\noindent $\gamma \ ::= $ \\
\hspace*{10mm} $| \quad \texttt{[}e...\texttt{]}$ \\
\hspace*{10mm} $| \quad \texttt{[}e...e\texttt{]}$ \\
\hspace*{10mm} $| \quad \texttt{[}e \ | \ x \texttt{ over } \gamma, \texttt{]}$ \\
\hspace*{10mm} $| \quad \texttt{[}e \ | \ x \texttt{ over } \gamma, e \texttt{]}$ \\

\noindent $\sigma \ ::= $ \\
\hspace*{10mm} $| \quad \texttt{int}$ \\
\hspace*{10mm} $| \quad \texttt{float}$ \\
\hspace*{10mm} $| \quad \texttt{bool}$ \\
\hspace*{10mm} $| \quad \texttt{char}$ \\

\noindent $\delta \ ::= $ \\
\hspace*{10mm} $| \quad \texttt{[]} $ \\
\hspace*{10mm} $| \quad \texttt{()} $ \\

\noindent $\tau \ ::= $ \\
\hspace*{10mm} $| \quad \sigma $ \\
\hspace*{10mm} $| \quad \delta \ \tau $ \\
\hspace*{10mm} $| \quad \tau \texttt{->} \tau $ \\

\noindent $\theta \ ::= $ \\
\hspace*{10mm} $| \quad x \ :: \  \tau $ \\
\columnbreak \\
$Entrypoint$ \\
main method \\

\noindent $Expressions$ \\
literals \\
variables \\
application \\
lambda abstraction \\
let binding \\
if-then-else \\
list comprehension \\

\noindent $List \ Comprehensions$ \\
infinite list \\
ranged list \\
parametric list \\
qualified list \\

\noindent $Proper \ Types$ \\
integer \\
floating point number \\
boolean \\
character \\

\noindent $Higher \ Order \ Type \ Constructors$ \\
list \\
tuple \\

\noindent $Types$ \\
proper type \\
higher order type \\
arrow type \\

\noindent $Type \ Annotations$ \\
type annotations \\
\end{multicols}

\newpage 
\subsection{Comments}
Comments in Rippl draw from Python and Haskell. Single line comments are marked by one pound sign, while multiline comments are delimited by curly braces and dashes. Both forms of commenting support nested comments.
\begin{lstlisting}[language=rippl]
{- The following code will first declare a variable with value 2
   After that, it will be raised to the power of 5 -}
   
let x = 2 in
2 ^ 5 # 32... what a #cool operation!

{- All programmers eventually fall victim to writing 
silly  {- superfluous -} comments, so it is important to 
learn how to write {- concise -} documentation! -}
\end{lstlisting}
\subsection{Keywords}
There are several keywords that are reserved in Rippl and thus not available for use as identifiers.
\begin{lstlisting}[language=rippl]
{- Data Types -}
# int, float, bool, char

{- Boolean Logic -}
# and, or, not, true, false

{- Program Structure -}
# over, let, in, fun, if, then, else, main

{- Higher Order Type Operators -}
# head, tail, cons, cat, len, first, sec
\end{lstlisting}
\subsection{Identifiers}
Identifiers can be any sequence of characters that start with either a letter or an underscore followed by any assortment of letters, numbers, and underscores.
\begin{lstlisting}[language=rippl]
my_identifier123        # Valid!
_my_other_indentifier456      # Also valid!
1_crazy_name          # What the heck? NO!
\end{lstlisting}
\newpage
\subsection{Operators}
The following subsections detail the various operators associated with each type in Rippl. It is worth noting that arithmetic expressions are evaluated using a mathematical
order of precedence with left associativity. That is, expressions are evaluated using the following rules in order of decreasing precedence: parentheses, exponents, unary operators, multiplication/division/modulus, addition/subtraction. 
\subsubsection{Integer Operators}
\begin{multicols}{3}
\noindent $Operator$ \\
\hspace*{5mm} + \\
\hspace*{5mm} - \\
\hspace*{5mm} * \\
\hspace*{5mm} / \\
\hspace*{5mm} \% \\
\hspace*{5mm} \^ \\
\hspace*{5mm} $>$ \\
\hspace*{5mm} $>=$ \\
\hspace*{5mm} $<$ \\
\hspace*{5mm} $<=$ \\
\hspace*{5mm} == \\
\hspace*{5mm} != \\
\columnbreak \\
\noindent $Type$ \\
$int \ \rightarrow \ int \ \rightarrow \ int $ \\
$int \ \rightarrow \ int \ \rightarrow \ int $ \\
$int \ \rightarrow \ int \ \rightarrow \ int $ \\
$int \ \rightarrow \ int \ \rightarrow \ int $ \\
$int \ \rightarrow \ int \ \rightarrow \ int $ \\
$int \ \rightarrow \ int \ \rightarrow \ int $ \\
$int \ \rightarrow \ int \ \rightarrow \ int $ \\
$int \ \rightarrow \ int \ \rightarrow \ int $ \\
$int \ \rightarrow \ int \ \rightarrow \ int $ \\
$int \ \rightarrow \ int \ \rightarrow \ int $ \\
$int \ \rightarrow \ int \ \rightarrow \ int $ \\
$int \ \rightarrow \ int \ \rightarrow \ int $ \\
\columnbreak \\
$Function$ \\
addition \\
subtraction \\
multiplication \\
division \\
modulus \\
power \\
greater than \\
greater than or equal \\
less than \\
less than or equal \\
equal \\
not equal \\
\end{multicols}
\subsubsection{Floating-point Operators}
Note that float operators differ from integer operators in that they require an additional '.' to aid in type inference.
\begin{multicols}{3}
\noindent $Operator$ \\
\hspace*{5mm} +. \\
\hspace*{5mm} -. \\
\hspace*{5mm} *. \\
\hspace*{5mm} /. \\
\hspace*{5mm} \^{}. \\
\hspace*{5mm} $>$. \\
\hspace*{5mm} $>=$. \\
\hspace*{5mm} $<$. \\
\hspace*{5mm} $<=$. \\
\hspace*{5mm} ==. \\
\hspace*{5mm} !=. \\
\columnbreak \\
\noindent $Type$ \\
$float \ \rightarrow \ float \ \rightarrow \ float $ \\
$float \ \rightarrow \ float \ \rightarrow \ float $ \\
$float \ \rightarrow \ float \ \rightarrow \ float $ \\
$float \ \rightarrow \ float \ \rightarrow \ float $ \\
$float \ \rightarrow \ float \ \rightarrow \ float $ \\
$float \ \rightarrow \ float \ \rightarrow \ float $ \\
$float \ \rightarrow \ float \ \rightarrow \ float $ \\
$float \ \rightarrow \ float \ \rightarrow \ float $ \\
$float \ \rightarrow \ float \ \rightarrow \ float $ \\
$float \ \rightarrow \ float \ \rightarrow \ float $ \\
$float \ \rightarrow \ float \ \rightarrow \ float $ \\
\columnbreak \\
$Function$ \\
addition \\
subtraction \\
multiplication \\
division \\
power \\
greater than \\
greater than or equal \\
less than \\
less than or equal \\
equal \\
not equal \\
\end{multicols}
\newpage
\subsubsection{Boolean Operators}
\begin{multicols}{3}
\noindent $Operator$ \\
\hspace*{5mm} \texttt{and} \\
\hspace*{5mm} \texttt{or} \\
\hspace*{5mm} \texttt{not} \\
\columnbreak \\
\noindent $Type$ \\
$bool \ \rightarrow \ bool \ \rightarrow \ bool $ \\
$bool \ \rightarrow \ bool \ \rightarrow \ bool $ \\
$bool \ \rightarrow \ bool $ \\
\columnbreak \\
\noindent $Function$ \\
and \\
or \\
negation \\
\end{multicols}
\subsubsection{Character Operators}
\begin{multicols}{3}
\noindent $Operator$ \\
\hspace*{5mm} == \\
\columnbreak \\
\noindent $Type$ \\
$char \ \rightarrow \ char \ \rightarrow \ bool $ \\
\columnbreak \\
\noindent $Function$ \\
equal \\
\end{multicols}
\subsubsection{List Operators}
\begin{multicols}{3}
\noindent $Operator$ \\
\hspace*{5mm} \texttt{cons} \\
\hspace*{5mm} \texttt{head} \\
\hspace*{5mm} \texttt{tail} \\
\hspace*{5mm} \texttt{len} \\
\hspace*{5mm} \texttt{cat} \\
\columnbreak \\
\noindent $Type$ \\
$a \ \rightarrow \ [a] \ \rightarrow \ [a] $ \\
$[a] \ \rightarrow \ a $ \\
$[a] \ \rightarrow \ [a] $ \\
$[a] \ \rightarrow \ int $ \\
$[a] \ \rightarrow \ [a] \ \rightarrow \ [a] $ \\
\columnbreak \\
\noindent $Function$ \\
construct \\
head \\
tail \\
length \\
concatenate \\
\end{multicols}
\subsubsection{Tuple Operators}
\begin{multicols}{3}
\noindent $Operator$ \\
\hspace*{5mm} \texttt{first} \\
\hspace*{5mm} \texttt{sec} \\
\columnbreak \\
\noindent $Type$ \\
$(a,b) \ \rightarrow \  a$ \\
$(a,b) \ \rightarrow \  b$ \\
\columnbreak \\
\noindent $Function$ \\
first element of tuple \\
second element of tuple \\
\end{multicols}

\subsection{Let Bindings}
Given the functional nature of Rippl, there are no imperative assignment statements. Instead, Rippl consists mainly of nested expressions strung together
using the let-in construct. This takes the form of \texttt{let identifier = expr1 in expr2}, where \texttt{indentifier} serves as a name binded to the value of \texttt{expr1} to be used in \texttt{expr2}. The value of a let binding chain is the value of the last expression.
\begin{lstlisting}[language=rippl]
let x = 1 in 
let y = 2 in 
let y = 3 in
x + y + z # 6
\end{lstlisting}

\subsection{If-then-else Expressions}
Like most programming languages, Rippl supports basic control flow through if-then-else expressions in the following form:
\texttt{if bool then expr else expr}. %TODO

\begin{lstlisting}[language=rippl]
let num = 5 in
let my_string = if num == 5 then "five" 
        else "not five" # five
\end{lstlisting}

\subsection{Lambda Abstractions}
One cornerstone of Rippl is to permit computations that require higher order functions, which is made possible by lambda abstractions. These anonymous functions are written using the \texttt{fun} keyword are \texttt{arrow type} expressions.
\begin{lstlisting}[language=rippl]
fun x -> x + 1 # int -> int
(fun x -> x + 1) 9 # 10 - application with an anonymous function
\end{lstlisting}
We can also bind these lambda abstractions to identifiers using a \texttt{let} expression. 
\begin{lstlisting}[language=rippl]
let add_one = fun x -> x + 1 in
add_one 9 # 10
\end{lstlisting}
Hope you have a sweet tooth! Rippl provides some syntactic sugar when it comes to name binding functions! The following expressions (after de-sugaring the syntax) are equivalent.
\begin{lstlisting}[language=rippl]
let add_one = fun x -> x + 1 # add_one :: int -> int
let cooler_add_one x = x + 1 # cooler_add_one :: int -> int
\end{lstlisting}
To top it all off, Rippl embraces the Haskell paradigm that all functions are curried. In other words, all functions in Rippl actually take just one parameter, either in arrow type form or proper/higher order type. This allows a Rippler to employ partial application on functions!
\begin{lstlisting}[language=rippl]
let sum_three x y z = x + y + z in
# sum_three :: int -> int -> int -> int
let sum_two_add_one = sum_three 1 in
#sum_two_add_one :: int -> int -> int
let add_four = sum_two_add_one 3 in 
# add_four :: int -> int
add_four 5  # 9
\end{lstlisting}
\pagebreak

\subsection{List Comprehensions}
List comprehensions are an elegant and concise way to define and construct a list. They use a well-known mathematical notation to substitute complex list operations like map and filter. \\

Rippl uses lazy evaluation for list comprehensions. This means that only the head of the list is initially fully evaluated and other elements are evaluated when their value is requested. \\

\begin{lstlisting}[language=rippl]
[1, 2, 3, 18, -1]
\end{lstlisting}


\subsubsection{Ranged List}
With ranged lists, we can create a list without specifying every element. Ranges will always have an interval of + 1 and are only allowed for integer lists. 

\begin{lstlisting}[language=rippl]
[18...24] # [18, 19, 20, 21, 22, 23, 24]

['a'...'z'] # not allowed

[1.2...2.7] # no sir, uncountable number of elements
\end{lstlisting}

\subsubsection{Parametric list}
Parametric lists allow users to specify a parameter.

\begin{lstlisting}[language=rippl]
[x * 2 | x over [3, 1, 4]] # [6, 2, 8] 

[x | x over [9...12]] # [9, 10, 11, 12]
\end{lstlisting}


\subsubsection{Qualified List}
With qualified lists, lists can be filtered by a condition.

\begin{lstlisting}[language=rippl]
[x ^ 2 | x over [1...10], x % 2 == 0] # [4, 16, 36, 64, 100]
\end{lstlisting}


\subsubsection{Parallel List}
Parallel list comprehensions are lists comprehensions with multiple parameters.

\begin{lstlisting}[language=rippl]
[x - y - z | x over [10, 1], y over [2, 3], z over [1, 5]] 
    # [7, 3, 6, 2, -2, -6, -3, -7]
    
[x + y | x over [10, 30, 50], y over [10...12], x != y]
    # [20, 21, 22, 40, 41, 42, 60, 61, 62]
\end{lstlisting}

\subsubsection{Infinite List}
Thanks to lazy evaluation, Rippl supports infinite lists. 

\begin{lstlisting}[language=rippl]
[3...] # [3, 4, 5, ...] 

[2, 22...24, 7...] # [2, 22, 23, 24, 7, 8, ...]
\end{lstlisting}
\section{Entrypoints and IO}
Like all pure functional languages, Rippl ensures that the evaluation of all expressions is free of side effects, or stateful interactions with the outside world. This prohibits the insertion of print statements inside functions that execute as they are evaluated. \\

Also as a pure functional language, Rippl guarantees that any computation given an argument will always return the same output. This means there can be no assumption of order of evaluation in a program as there is in the imperative paradigm. This prohibits the use of language constructs like \texttt{input} or \texttt{scanf} that read and return a value read from the user with no arguments (save the format-friendly strings present; this is more like a separate print/IO action than a true argument). With no varying argument, the function call in a pure language should return the same value each time which should not be the behavior of an IO operation. \\

Pure functional languages like Haskell get around this by giving all the IO function calls a hidden argument and strings each call together as arguments and dependencies to create a proper ordering by using an IO monad. This operation is sugared up into an imperative-looking \texttt{do} construct in their main methods. \\

Like in Haskell, the main method in Rippl is the top-level entrypoint into a program. In order to avoid the use of higher-kinded types like in Haskell, the main method is used such that there is only one input operation allowed and one output operation allowed, strictly evaluated in that order.  \\

User input is represented as the argument to the main function. It can be bound as a named argument that is then used for evaluation of the body of the main function if it requires user input. \\

\begin{lstlisting}[language=rippl]
# this main method takes user input and stores it in `arg`
main arg = ... 
\end{lstlisting}

Otherwise, to signify that nothing is to be read from user input, an underscore takes the place of the first argument (much like the wildcard in Haskell). \\

\begin{lstlisting}[language=rippl]
# this main method takes no user input
main _ = ...    
\end{lstlisting}

The evaluated value of the body of the main function is printed. The "Hello, world!" program would be written as follows in Rippl.

\begin{lstlisting}[language=rippl]
# this prints "Hello, world!" to the terminal
main _ = "Hello, world!"
\end{lstlisting}

If the right-hand side of the main method were a more complex computation expressed in terms of other function definitions provided in the file, the full expression would be evaluated and printed. This means that the main function is also fully polymorphic in terms of its input and output. \\

\begin{lstlisting}[language=rippl]
main :: a -> b
\end{lstlisting}

It might be noted that this construct of IO in a language doesn't allow for a function not to have output (Rippl doesn't have a \texttt{void} value representing a bottom type). However, this design should be reasonable since Rippl currently supports no other output operations so programs would be moot otherwise. \\
\section{Sample Programs}
\subsection{Infinite Sum}
This program returns the sum of the first $n$ positive integers using an 
infinite list.
\begin{lstlisting}[language=rippl]
inf_sum :: int -> int
inf_sum n = let infinity = [1...] in
    let rec_inf_sum x list acc = if x == 0 or (len list) == 0
        then acc
        else rec_inf_sum (x-1) (tail list) ((head list) + acc)
    in rec_inf_sum n infinity 0
\end{lstlisting}
\subsection{Collatz Conjecture}
The Collatz conjecture is a conjecture in mathematics surrounding the iterative 
function shown below:
$$ \begin{cases} 
      \frac{1}{2}x & x \texttt{ is even} \\
      3x + 1 & x \texttt{ is odd} 
   \end{cases}
$$
Lothar Collatz proposed that this sequence will always converge to $1$ starting 
from an arbitrary positive integer. Given a starting integer, the following 
program returns a list of iterations that ends in 1. The program hence uses $1$ 
as its base case even though it hasn't been mathematically proven that all 
numbers will eventually reach this case, but we believe in Lothar.
\begin{lstlisting}[language=rippl]
collatz :: int -> [int]
collatz n =
    let rec_collatz n list =
        if n == 1
            then 1 cons list
        else if n % 2 == 0
            then rec_collatz (n / 2) (n cons list)
            else rec_collatz (3*n +1) (n cons list)
    in rec_collatz n []
\end{lstlisting}
\subsection{Weak Prime Number Theorem}
Bertrand's Postulate stated in the Weak Prime Number Theorem that there is 
always a prime number to be found between some $n$ and its double $2n$. This 
postulate was later proven by Pafnuty Chebyshev and refined by Paul Erd\"{o}s. 
The following program includes a function that determines the primality of a 
number and a function that takes a $n$ and returns the first prime between 
$n$ and $2n$.
\begin{lstlisting}[language=rippl]
is_prime :: int -> bool
is_prime n =
    let max = n / 2 in
    let range = [2...max] in
    let divisors = [x | x over range, n % x == 0] in
    len divisors == 0
    
prime_number_theorem :: int -> int
prime_number_theorem n =
    let range = [(n+1)...2*n] in
    let odd_range = [x | x over range, x % 2 != 0 ] in
    foldl (fun prev -> fun curr -> if is_prime then prev else curr)
        (head odd_range) odd_range
\end{lstlisting}
\subsection{Entrypoint and IO}
This program demonstrates how a main method would be written to take in a user-inputted number and print out the full list of iterations generated by the formula in the Collatz Conjecture.
\begin{lstlisting}[language=rippl]
main :: int -> [int]
main n = collatz n      # 10 -> [10,5,16,8,4,2,1]
\end{lstlisting}
\end{document}
